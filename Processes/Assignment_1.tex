\documentclass[16pt]{amsart}           % use "amsart" instead of "article" for AMSLaTeX format
\usepackage{geometry}                                % See geometry.pdf to learn the layout options. There are lots.
\geometry{a3paper}                                   % ... or a4paper or a5paper or ...
%\geometry{landscape}                                % Activate for rotated page geometry
\usepackage[parfill]{parskip}                    % Activate to begin paragraphs with an empty line rather than an indent
\usepackage{graphicx}                                % Use pdf, png, jpg, or eps§ with pdflatex; use eps in DVI mode
                                % TeX will automatically convert eps --> pdf in pdflatex
\usepackage{amssymb}

%SetFonts
\usepackage{amsmath}
\usepackage{hyperref}
%SetFonts
\newcommand{\me}{\author{Abhay Shankar K : cs21btech11001}
\maketitle}

\begin{document}

\title{CS3510 : Assignment 1}
\me
    \begin{enumerate}
        \item \begin{enumerate}
            \item Describe a mechanism for enforcing memory protection in order to prevent a program
            from modifying the memory associated with other programs.

            \begin{itemize}
                \item Dynamic tainting: Durind dynamic memory allocation, 
                both the memory and the corresponding pointer are 'tainted' using the same 
                'taint mark'. Taint marks are then suitably propagated while the program 
                executes and are checked every time a memory address m is accessed through 
                a pointer p; if the taint marks associated with m and p differ, the 
                execution is stopped and the illegal access is reported.
                \item Paged virtual memory: Given an application, unallocated pages and pages allocated to any other application, do not have
                any addresses from the application point of view. Every memory address either points to a page allocated to that 
                application, or generates an interrupt called a page fault. 


            \end{itemize}


\item While memory protection is necessary, there are several examples where programs
require memory to be shared between them. Please give two such examples.

\begin{itemize}
    \item Shared memory facilitates inter-process communication (IPC), i.e. a way of exchanging data between programs running at the same time.
    One process will create an area in RAM which other processes can access.
    \item Shared memory can be used to conserve memory space by directing accesses to copies of a piece of data to a
    single instance instead.
\end{itemize}
Source : \href{https://en.wikipedia.org/wiki/Shared_memory}{Wikipedia}
        \end{enumerate} 
        % \newpage
        \item Please study the PCB (Process Control Block) structure of a Linux system from any reliable
        Internet source. Describe any 10+ fields in the PCB of a process in the latest Linux Operating
        System. Please cite your source as well like in question 2.

    \begin{verbatim}
unsigned int              __state;            ::The process state::
int                       prio;               ::Process/thread priority::
struct sched_statistics   stats;              ::Scheduling statistics::
int                       pdeath_signal;      ::The signal sent when the parent dies::
pid_t                     pid;                ::Thread ID from kernel's POV::
pid_t                     tgid;               ::Thread group ID equivalent to POSIX process-ID::
struct task_struct __rcu  *real_parent;       ::Real parent process::
struct list_head          children;           ::List of child processes::
struct list_head          sibling;            ::List of sibling processes::
struct task_struct        *group_leader;      ::Leader of thread group::
struct list_head          ptraced;            ::list of tasks this task is using ptrace() on.::
struct list_head          thread_group;       ::list of threads of this process::
unsigned long             nvcsw;              ::#Voluntary Context switch counts::
unsigned long             nivcsw;             ::#Involuntary Context switch counts::
u64                       start_time;         ::Monotonic time in ns::
u64                       start_boottime;     ::Boot based time in ns::
char                      comm[TASK_COMM_LEN];::executable name, excluding path, like ls::
struct fs_struct          *fs;                ::Filesystem information::
struct files_struct       *files;             ::Open file information::
\end{verbatim}

        Sources : \begin{itemize}
            \item \href{https://github.com/torvalds/linux/blob/master/include/linux/sched.h}{GitHub}
            \item \href{https://stackoverflow.com/questions/9305992/if-threads-share-the-same-pid-how-can-they-be-identified}{Stack Overflow}
        \end{itemize}
    \end{enumerate}
\end{document}