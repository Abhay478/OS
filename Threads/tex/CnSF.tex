\documentclass{amsart}
\usepackage{amsmath}
\usepackage{graphicx}
\usepackage{amssymb}
\usepackage{longtable}
\usepackage{listings}
\usepackage{tikz}
\usetikzlibrary{positioning}

\newcommand{\me}{
    \author{Abhay Shankar K: cs21btech11001}
    \maketitle
}
\begin{document}
\title{Theory Assignment 2}
\me

% \tikzset{gray box/.style={fill=gray!20,
%         draw=gray,
%         minimum width = {3*#1ex},
%         minimum height = {2em}
%         },
%         period/.style={
%         minimum width = {3*#1ex},
%         minimum height = {2em},
%         anchor=west
%         },
%         annotation/.style={anchor=north},
%         zlabel/.style={anchor=south}}
\newcommand{\brak}[1]{\ensuremath{\left(#1\right)}} % brackets
\newcommand{\abrak}[1]{\ensuremath{\left\langle#1\right\rangle}}
\newcommand{\sbrak}[1]{\ensuremath{\left[#1\right]}}
\newcommand{\cbrak}[1]{\ensuremath{\left\{#1\right\}}}
    \begin{enumerate}
        \item 6.8
        
            \begin{lstlisting}[escapechar=@]
1                void bid(double amount) { 
2                    if (amount > highestBid)
3                        highestBid = amount;
4                }
            \end{lstlisting}

            A race condition can be caused when two \texttt{bid()} calls and let occur simultaneously, and both \texttt{amount} in both cases is greater than \texttt{highest bid}.
            Let these calls be \(c_1\) and \(c_2\), \(c_1.amount > c_2.amount\). So if \(c_1\) enters the \texttt{if} block in 2 first, and then \(c_2\) enters. 
            Then \(c_1\) executes 3, setting \texttt{highestBid} to the correct value, \(c_1.amount\). Lastly, \(c_2\) executes 3 (which it ought to, since it has already entered the \texttt{if} block), setting the value of \texttt{highestBid} to \(c_2.amount\).

            So, race conditions cause problems.

        \item 6.11
            
            Yes, the given `compare and CAS' idiom is viable. Since any number of threads can read a shared variable, the surrounding \texttt{if} block does not impact the efficacy of the algorithm.
            Since the \texttt{compare\_and\_swap()} function is still called within the \texttt{if} block, no two threads can acquire the lock simultaneously.
            The extra branch instruction will preserve mutual exclusion.

        \item 6.23
        
            An implementation of a limited open-socket-count with semaphores is fairly straightforward.
            We can create a semaphore initialised to N, and the code for accepting a new connection 
            can be placed between a \texttt{sem\_wait()} call and a \texttt{sem\_post()} call. 
            If more requests arrive, they will be blocked by the wait call.

        \item Reader - writer.
            
            \begin{lstlisting}[escapechar=@]

1   semaphore rw_mutex = 1; 
2   semaphore mutex = 1; 
3   int read_count = 0;
4   int history = 0;
5       
6       writer() {
7           while (true) { 
8               wait(rw_mutex);
9               history = 0;
10              ...
11              /* writing is performed */
12              ... 
13              signal(rw_mutex);
14          }
15      }
16
17      reader() {
18          while (true) { 
19              while(history > 20); // keeps incoming threads occupied, 
                            ensures that semaphore released in 34 is not 
                            taken up by a reader > 20 times in a row.
20              wait(mutex);
21              read_count++;
22              history++;
23              if (read_count == 1 || history == 1)
24                  wait(rw_mutex); 
25              signal(mutex);
26
27              ...
28              /* reading is performed */
29              ... 
30              
31              wait(mutex);
32              read_count--;
33              if (read_count == 0 || history > 20) // too many readers 
                                                    executed in a row, 
                                                    gives writer a chance.
34                  signal(rw_mutex); 
35              signal(mutex);
36          }
37      }
        \end{lstlisting}    

        If twenty readers acquire the lock in a row, then further requests by readers are blocked 
        (busy wait), and once all active readers terminate, a writer (since no readers are waiting on 
        the lock) will enter. It then resets the \texttt{history} variable, allowing readers to make 
        requests for the lock again. Once the writer finishes, any one of the reader/writer threads 
        may acquire the lock.

        \item Atomic increment - using critical section. Starvation-freedom ensured through bounded waiting.
        
            \begin{lstlisting}[escapechar=@]
    
1   int incr(int n) {
2       while (true) {
3           waiting[i] = true;
4           key = 1;
5           while (waiting[i] && key == 1)
6               key = compare and swap(&lock,0,1); 
7           waiting[i] = false;
8           n++;
9           j = (i + 1) % n;
10          while ((j != i) && !waiting[j])
11              j = (j + 1) % n;
12              if (j == i)
13                  lock = 0;
14              else
15                  waiting[j] = false;
16      }
17  }
            \end{lstlisting}
            
    \end{enumerate}


\end{document}