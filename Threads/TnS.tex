\documentclass{amsart}
\usepackage{amsmath}
\usepackage{graphicx}
\usepackage{amssymb}
\usepackage{longtable}
\usepackage{tikz}
\usetikzlibrary{positioning}

\newcommand{\me}{
    \author{Abhay Shankar K: cs21btech11001}
    \maketitle
}
\begin{document}
\title{Theory Assignment 1}
\me

\tikzset{gray box/.style={fill=gray!20,
        draw=gray,
        minimum width = {3*#1ex},
        minimum height = {2em}
        },
        period/.style={
        minimum width = {3*#1ex},
        minimum height = {2em},
        anchor=west
        },
        annotation/.style={anchor=north},
        zlabel/.style={anchor=south}}
\newcommand{\brak}[1]{\ensuremath{\left(#1\right)}} % brackets
\newcommand{\abrak}[1]{\ensuremath{\left\langle#1\right\rangle}}
\newcommand{\sbrak}[1]{\ensuremath{\left[#1\right]}}
\newcommand{\cbrak}[1]{\ensuremath{\left\{#1\right\}}}
    \begin{enumerate}
        \item Given \[Prioritynumber = \brak{base + \frac{\text{recent CPU usage } U}{2}}\] where higher the priority, lower the priority number, and \begin{itemize}
            \item base = 60
            \item U for P1 = 40
            \item U for P2 = 18
            \item U for P3 = 10
        \end{itemize}
        What will be their new priorities? Furthermore, does the scheduler raise or lower the relative priority of a CPU bound process?

        The new priorities: \begin{itemize}
            \item P1: $60 + \frac{40}{2} = 80$ \\
            
            \item P2: $60 + \frac{18}{2} = 69$ \\
            
            \item P3: $60 + \frac{10}{2} = 65$ \\
        \end{itemize}

        Clearly, U is high for a CPU bound process. Since the prioriity number is linear in U, the scheduler decreases the priority of CPU bound processes.

        \item Question 4.19 from the 10th edition of the book on the page EX-9 of the book
            
        Output:

                CHILD: value = 5PARENT: value = 0

            \begin{itemize}
                \item Line C: ``CHILD: value = 5''
                \item Line P: ``PARENT: value = 0''
            \end{itemize}

        \item Question 5.17 from the 10th edition of the book on the page EX-13 of the book
            
        \begin{tabular}{|c|c|c|}
                \hline
                Process & Burst time & Priority \\
                \hline
                $P_1$ & 5 & 4 \\
                $P_2$ & 3 & 1 \\
                $P_3$ & 1 & 2 \\
                $P_4$ & 7 & 2 \\
                $P_5$ & 4 & 3 \\ \hline 
            \end{tabular}

            \begin{enumerate}
                \item Draw four Gantt charts that illustrate the execution of these pro- cesses using the following scheduling algorithms: FCFS, SJF, non-preemptive priority (a larger priority number implies a higher priority), and RR (quantum = 2).
                    \begin{itemize}
                        \item FCFS:
                        \bigskip

                        \begin{tikzpicture}[node distance=-0.7pt]
                            \node [gray box=5] (P1) {\(P_1\)};
                            \node [gray box=3, right=of P1] (P2) {\(P_2\)};
                            \node [gray box=1, right=of P2] (P3) {\(P_3\)};
                            \node [gray box=7, right=of P3] (P4) {\(P_4\)};
                            \node [gray box=4, right=of P4] (P5) {\(P_5\)};

                            \node [annotation] at (P1.south west) {0};
                            \node [annotation] at (P1.south east) {5};
                            \node [annotation] at (P2.south east) {8};
                            \node [annotation] at (P3.south east) {9};
                            \node [annotation] at (P4.south east) {16};
                            \node [annotation] at (P5.south east) {20};
                        \end{tikzpicture}
                        \item SJF: 
                        \bigskip


                        \begin{tikzpicture}[node distance=-0.7pt]
                            \node [gray box=1] (P1) {\(P_3\)};
                            \node [gray box=3, right=of P1] (P2) {\(P_2\)};
                            \node [gray box=4, right=of P2] (P3) {\(P_5\)};
                            \node [gray box=5, right=of P3] (P4) {\(P_1\)};
                            \node [gray box=7, right=of P4] (P5) {\(P_4\)};

                            \node [annotation] at (P1.south west) {0};
                            \node [annotation] at (P1.south east) {1};
                            \node [annotation] at (P2.south east) {4};
                            \node [annotation] at (P3.south east) {8};
                            \node [annotation] at (P4.south east) {13};
                            \node [annotation] at (P5.south east) {20};
                        \end{tikzpicture}
                        \item NPP: 
                        \bigskip


                        \begin{tikzpicture}[node distance=-0.5pt]
                            \node [gray box=5] (P1) {\(P_1\)};
                            \node [gray box=4, right=of P1] (P2) {\(P_5\)};
                            \node [gray box=1, right=of P2] (P3) {\(P_3\)};
                            \node [gray box=7, right=of P3] (P4) {\(P_4\)};
                            \node [gray box=3, right=of P4] (P5) {\(P_2\)};

                            \node [annotation] at (P1.south west) {0};
                            \node [annotation] at (P1.south east) {5};
                            \node [annotation] at (P2.south east) {9};
                            \node [annotation] at (P3.south east) {10};
                            \node [annotation] at (P4.south east) {17};
                            \node [annotation] at (P5.south east) {20};
                        \end{tikzpicture}
                        \item RR: 
                         \bigskip


                        \begin{tikzpicture}[node distance=-0.7pt]
                            \node [gray box=2] (P1) {\(P_1\)};
                            \node [gray box=2, right=of P1] (P2) {\(P_2\)};
                            \node [gray box=1, right=of P2] (P3) {\(P_3\)};
                            \node [gray box=2, right=of P3] (P4) {\(P_4\)};
                            \node [gray box=2, right=of P4] (P5) {\(P_5\)};
                            \node [gray box=2, right=of P5] (P6) {\(P_1\)};
                            \node [gray box=1, right=of P6] (P7) {\(P_2\)};
                            \node [gray box=2, right=of P7] (P8) {\(P_4\)};
                            \node [gray box=2, right=of P8] (P9) {\(P_5\)};
                            \node [gray box=1, right=of P9] (P10) {\(P_1\)};
                            \node [gray box=3, right=of P10] (P11) {\(P_4\)};

                            \node [annotation] at (P1.south west) {0};
                            \node [annotation] at (P1.south east) {2};
                            \node [annotation] at (P2.south east) {4};
                            \node [annotation] at (P3.south east) {5};
                            \node [annotation] at (P4.south east) {7};
                            \node [annotation] at (P5.south east) {9};
                            \node [annotation] at (P6.south east) {11};
                            \node [annotation] at (P7.south east) {12};
                            \node [annotation] at (P8.south east) {14};
                            \node [annotation] at (P9.south east) {16};
                            \node [annotation] at (P10.south east) {17};
                            \node [annotation] at (P11.south east) {20};

                        \end{tikzpicture}

                    \end{itemize}
                    
                \item What is the turnaround time of each prcess for each algorithm?
                    \begin{enumerate}
                    \item FCFS: \begin{enumerate}
                        \item P1: 5
                        \item P2: 8
                        \item P3: 9
                        \item P4: 16
                        \item P5: 20
                    \end{enumerate}
                    \item SJF: \begin{enumerate}
                        \item P1: 13
                        \item P2: 4
                        \item P3: 1
                        \item P4: 20
                        \item P5: 8
                    \end{enumerate}
                    \item NPP: \begin{enumerate}
                        \item P1: 5
                        \item P2: 20
                        \item P3: 10
                        \item P4: 17
                        \item P5: 9
                    \end{enumerate}
                    \item RR: \begin{enumerate}
                        \item P1: 17
                        \item P2: 12
                        \item P3: 5
                        \item P4: 20
                        \item P5: 16
                    \end{enumerate}
                \end{enumerate}
                \item What is the waiting time of each process for each of these scheduling algorithms?
                    \begin{enumerate}
                    \item FCFS: \begin{enumerate}
                        \item P1: 0
                        \item P2: 5
                        \item P3: 8
                        \item P4: 9
                        \item P5: 16
                    \end{enumerate}
                    \item SJF: \begin{enumerate}
                        \item P1: 8
                        \item P2: 1
                        \item P3: 0
                        \item P4: 13
                        \item P5: 4
                    \end{enumerate}
                    \item NPP: \begin{enumerate}
                        \item P1: 0
                        \item P2: 17
                        \item P3: 9
                        \item P4: 10
                        \item P5: 5
                    \end{enumerate}
                    \item RR: \begin{enumerate}
                        \item P1: 12
                        \item P2: 9
                        \item P3: 4
                        \item P4: 13
                        \item P5: 12
                    \end{enumerate}
                \end{enumerate}
                \item Which of the algorithms results in the minimum average waiting time (over all processes)?
                    Average waiting times: \begin{enumerate}
                        \item FCFS: 7.6
                        \item SJF: 5.2
                        \item NP: 8.2
                        \item RR: 10
                    \end{enumerate}

                    Clearly, SJF results in minimum average waiting time.

            
               \end{enumerate}

        \item Question 5.24 from the 10th edition of the book on the page EX-14 of the book
        \begin{enumerate}
            \item FCFS: A running process's priority increases faster than any ready ones and always remains nonnegative, and therefore, there is no preemption. Processes execute in order of arrival.
            \item LIFO: The priority can never increase, and so the highest priority process is one that arrived most recently. Moreover, \(\beta > \alpha\), so a pocess can only be preempted and placed in the ready queue (or rather, ready stack) upon the arrival of a new process, and will even then have he highest priority among all waiting processes (i.e. top of stack). 
        \end{enumerate}
        \item Question 5.35 from the 10th edition of the book on the page EX-16 of the book
            Consider processes $P_1$ and $P_2$, where $p_1 = 50$, $t_1 = 25$ and $p_2 = 75$, $t_2 = 30$.   
            Let \(\| P_i \)  denote the preemptive shift of a process $P_i$ to the ready queue.

        \begin{enumerate}
                \item Rate-monotonic scheduling: Not possible.
                    
                % \begin{ganttchart}[hgrid, vgrid, inline,
                %             milestone inline label node/.append style={left=5mm}]{1}{16}
                %             \gantttitle{NPP}{16} \\
                %             \gantttitlelist{0, 5, ..., 75}{1}\\
                %             \ganttbar[bar height = 0.5]{$P_1$}{1}{5}  
                %             \ganttbar[bar height = 0.5]{$P_2$}{6}{10}
                %             \ganttbar[bar height = 0.5]{$P_1$}{11}{15}
                %             % \ganttbar[bar height = 0.3]{$P_1$}{}{}
                %         \end{ganttchart}

                \begin{tikzpicture}[node distance=-0.5pt]
                    \node [gray box=5] (P1) {\(P_1\)};
                    \node [gray box=5, right=of P1] (P2) {\(P_2\)};
                    \node [gray box=5, right=of P2] (P3) {\(P_1\)};

                    \node [annotation] at (P1.south west) {\(\uparrow P_1\), \(P_2\)};
                    \node [annotation] at (P1.south east) {\(\downarrow P_1\)};
                    \node [annotation] at (P2.south east) {\(\uparrow P_1 \| P_2\)};
                    \node [annotation] at (P3.south east) {\(P_2\) failed};

                    \node [zlabel] at (P1.north west) {0};
                    \node [zlabel] at (P1.north east) {25};
                    \node [zlabel] at (P2.north east) {50};
                    \node [zlabel] at (P3.north east) {75};
                \end{tikzpicture}

                \item EDF scheduling: Possible.
                \raggedright
                    \begin{tikzpicture}[node distance=-0.5pt]
                        \node [gray box=5] (P1) {\(P_1\)};
                        \node [gray box=6, right=of P1] (P2) {\(P_2\)};
                        \node [gray box=5, right=of P2] (P3) {\(P_1\)};
                        \node [gray box=6, right=of P3] (P4) {\(P_2\)};
                        \node [gray box=5, right=of P4] (P5) {\(P_1\)};
                        \node [gray box=3, right=of P5] (P6) {\(-\)};
                        
                        \node [period=10] (Q1) at (P1.west){};
                        \node [period=10, right=of Q1] (Q2) {};
                        \node [period=10, right=of Q2] (Q3) {};

                        \node [period=15] (R1) at (P1.west){};
                        \node [period=15, right=of Q2] (R2) {};

                    \node [annotation] at (P1.south west) {\(\uparrow P_1, P_2\)};
                    \node [annotation] at (P1.south east) {\(\downarrow P_1\)};
                    \node [annotation] (a1) at (Q1.south east) {50};
                    \node [annotation] at (a1.south) {\(\uparrow P_1\)};
                    \node [annotation] at (P2.south east) {.    \(\downarrow P_2\)};
                    \node [annotation] at (P3.south east) {.    \(\downarrow P_1\)};
                    \node [annotation] (b1) at (R1.south east) {75};
                    \node [annotation] at (b1.south) {\(\uparrow P_2\)};
                    \node [annotation] (a2) at (Q2.south east) {100};
                    \node [annotation] at (a2.south) {\(\uparrow P_1\)};
                    \node [annotation] at (P4.south east) {\(\downarrow P_2\)};
                    \node [annotation] at (P5.south east) {\(\downarrow P_1\)};
                    \node [annotation] at (P6.south east) {\(\uparrow P_1, P_2\)};
                    

                    \node [zlabel] at (P1.north west) {0};
                    \node [zlabel] at (P1.north east) {25};
                    \node [zlabel] at (P2.north east) {55};
                    \node [zlabel] at (P3.north east) {80};
                    \node [zlabel] at (P4.north east) {110};
                    \node [zlabel] at (P5.north east) {135};
                    \node [zlabel] at (P6.north east) {150};
                \end{tikzpicture}

                The above 150 time units repeat forever, and the processor remains idle during the last 15 time units.
            \end{enumerate}
    \end{enumerate}


\end{document}